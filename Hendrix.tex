
 \subsection{Constance Hendrix}
Greetings.  My name is Constance Hendrix.  I go-by Constance.  I’m a wife, an electrical engineer, a 30-year Air Force veteran, and a current Engineering Security PhD student here at UCCS (see Canvas picture in Figure 1. 
\begin{figure}[h]
    \centering
    \includegraphics[scale=0.06]{hendrix_canvas.jpg}
    \label{fig:me}
    \caption{Constance, about to be fed, while on a overdue retreat!}
\end{figure}
I’m coming in with a Masters of Science degree in electrical engineering with a focus in navigation systems from the Air Force Institute of Technology and a Masters of Business Administration from the University of West Florida.  I'm also a licensed electrical engineer in the state of Colorado and a certified Project Management Professional. My overarching goal for graduate school is to make a significant contribution to my engineering field and lay the foundation for a future in academia.  My research interests include reliable and accurate navigation, secure satellite communications, biologically-influenced design, artificial intelligence, and signal processing.  I have yet to specifically identify my security research topic, but know it will include artificial intelligence, signal processing, and/or edge computing. I hope to select a topic by next semester and will be preparing for my oral qualifier next Spring as well.  My goal for this course is to narrow my research focus for my degree, learn new and efficient ways to conduct research, and develop research questions. Outside of school, I work part-time as a position, navigation and timing (PNT) engineer, enjoy reading, working on stained glass creations, quilting, camping, fly fishing, hiking, cooking, and gardening.  Quilting is a tradition for the women in my family.  Even though I just recently started, I am excited to carry on this tradition. \\

\textbf{Question from Jackie}
Hi Constance, you have a very impressive knowledge background. I do not know a lot about navigation systems, but I always have a question on how navigation works when two teams are digging an undersea tunnel toward each other and they align perfectly? \\

\textbf{Answer to Jackie}:  Hi Jackie!  Good question.  Most of my knowledge assumes satellite navigation is available.  Without Googling, one possible option would be to map out the two locations via GPS, calculating relative bearing.  Once you have the bearings you can begin.  There will probably be a random drift if you used a magnetic compass, which would increase the possibility that the two boring machines won't meet up.  There would have to be intermediate checks to make sure they don't drift off course.  This is where my research would have to begin.  \\

I've never been asked this  -- I'm sure someone has identified an optimal method.  Interesting to think about.  Thanks!   \\  

\textbf{Question from Jordan}
Have you used git before? Most EEs I know end up in software at some point. I've only used Subversion before this year so git is still new to me.  \\

\textbf{Answer to Jordan}:  Yes, I use Git at work.  I push up code on a regular basis to my team's repo.  I've also used GitLab in a previous job (program manager).  I only know the basics.  Actually, I think I'm a rare case.  I've been an engineer for about 20 years.  I've been suck in program management and leadership positions during most of that time (military officer).  Retired last year, so given this new freedom, I decided to get back to my roots.  I have only been doing real engineering since Feb, so I'm somewhat of a novice.  Never heard of Subversion... will have to look it up.  Thanks for the question!

\textbf{Question from Ntumpha}
Hi Constance, am here to congratulate you on your services to the country. You are an inspiration. My question is after 30 years, how was your transition from military life to civilian life?
